\section{Attention}
\label{sec:attention}


\subsection{Attention layers in neural networks}
\label{sec:attention-what}

\begin{figure}[tb]
    \centering
    % \documentclass{standalone}
% \usepackage{tikz}
% \usetikzlibrary{positioning,fit,calc,arrows.meta,backgrounds}
% \usepackage[outline]{contour}
% \begin{document}
\tikzset{
    base/.style = {font=\small}, % \small\sffamily
    box/.style = {base, rectangle, rounded corners, draw=black,
                   minimum width=0.1cm, minimum height=0.5cm, 
                   text centered},
    input/.style = {box},
    lstm/.style = {box, minimum width=1.5cm},
    feature/.style = {box,
                   minimum height=6mm,},
    group/.style = {box, dashed},
    labels/.style = {box, minimum width=1.5cm},
    description/.style = {text width=4cm},
    >=LaTeX
}
\begin{tikzpicture}[
    node distance=1.5cm,
    align=center % necessary for intra-node linebreaks
    ]

  \node (zT) [feature] {$z_T$};
  \node (zdot) [left=10mm of zT] {$...$};
  \node (z2) [feature, left=10mm of zdot] {$z_2$};
  \node (z1) [feature, left=10mm of z2] {$z_1$};
  \node (embed) [description, left=10mm of z1] {embeddings};
  
%   \node (h1fwd) [feature, above=5mm of z1] {$\overrightarrow{h}_1$};
%   \node (h2fwd) [feature, above=5mm of z2] {$\overrightarrow{h}_2$};
%   \node (hdotfwd) [right=9mm of h2fwd] {$...$};
%   \node (hTfwd) [feature, above=5mm of zT] {$\overrightarrow{h}_T$};
%   \node (lstm) [description, above=7mm of embed] {bidirectional\\RNN};
  
%   \node (h1bwd) [feature, above=2mm of h1fwd] {$\overleftarrow{h}_1$};
%   \node (h2bwd) [feature, above=2mm of h2fwd] {$\overleftarrow{h}_2$};
%   \node (hdotbwd) [right=9mm of h2bwd] {$...$};
%   \node (hTbwd) [feature, above=2mm of hTfwd] {$\overleftarrow{h}_T$};

%   \begin{scope}[on background layer]
%       \node (h1) [group, fit={(h1fwd) (h1bwd)}] {};
%       \node (h2) [group, fit={(h2fwd) (h2bwd)}] {};
%       \node (hT) [group, fit={(hTfwd) (hTbwd)}] {};
%   \end{scope}

  \node (h1) [feature, above=11mm of z1] {$h_1$};
  \node (h2) [feature, above=11mm of z2] {$h_2$};
  \node (hdot) [above=14mm of zdot] {$...$};
  \node (hT) [feature, above=11mm of zT] {$h_T$};
  \node (encoder) [description, left=10mm of h1] {encoder};
  
%   \node (v) [feature, above right=0mm and 15mm of hTbwd] {$v$};
    \node (v) [feature, above right=-2mm and 15mm of hT] {$v$};


  \node (alpha1) [feature, above=10mm of h1] {$\alpha_1$};
  \node (alpha2) [feature, above=10mm of h2] {$\alpha_2$};
%   \node (alphadot) [above=14mm of hdotbwd] {$...$};
  \node (alphadot) [above=13mm of hdot] {$...$};
  \node (alphaT) [feature, above=10mm of hT] {$\alpha_T$};
  \node [description, left=10mm of alpha1] {attention weights};
  
  \node (halpha1) [feature, above=5mm of alpha1] {$h_{\alpha_1}$};
  \node (halpha2) [feature, above=5mm of alpha2] {$h_{\alpha_2}$};
  \node (halphadot) [above=8mm of alphadot] {$...$};
  \node (halphaT) [feature, above=5mm of alphaT] {$h_{\alpha_T}$};
  \node (attnO) [description, left=10mm of halpha1] {attention output};
  
%   \node (dense) [feature, above=40mm of hdotbwd] {dense layer};
  \node (dense) [feature, above=40mm of hdot] {dense layer};
  \node (yhat) [above=5mm of dense] {$\hat{y}$};
  \node [description, above=9mm of attnO] {decoder};

  
  \draw[->] (z1) -- (h1);
  \draw[->] (z2) -- (h2);
  \draw[->] (zT) -- (hT);
  
%   \draw[->] (h1fwd) -- (h2fwd);
%   \draw[->] (h2fwd) -- (hdotfwd);
%   \draw[->] (hdotfwd) -- (hTfwd);
  
%   \draw[->] (h2bwd) -- (h1bwd);
%   \draw[->] (hdotbwd) -- (h2bwd);
%   \draw[->] (hTbwd) -- (hdotbwd);
  
  \draw[->] (v) -- (alpha1.south);
  \draw[->] (v) -- (alpha2.south);
  \draw[->] (v) -- (alphaT.south);

  \draw[->] (h1) -- (alpha1);
  \draw[->] (h2) -- (alpha2);
  \draw[->] (hT) -- (alphaT);
  
  \draw[->] (h1.north) to[out=150,in=210] (halpha1.south);
  \draw[->] (h2.north) to[out=150,in=210] (halpha2.south);
  \draw[->] (hT.north) to[out=150,in=210] (halphaT.south);
  
  \draw[->] (alpha1) -- (halpha1);
  \draw[->] (alpha2) -- (halpha2);
  \draw[->] (alphaT) -- (halphaT);


  \draw[->] (halpha1) -- (dense.south);
  \draw[->] (halpha2) -- (dense.south);
  \draw[->] (halphaT) -- (dense.south);
  \draw[->] (dense) -- (yhat);
  
  
\end{tikzpicture}
% \end{document}

    \caption
    [Neural network with attention layer]
    {The neural classification model encodes token embeddings $z$ with a neural network.
    Multiplying the resulting matrix $h$ with the context vector $v$ creates token-wise attention weights $\alpha$, based on which the weighted token representations $h$\textsubscript{$\alpha$} are generated.}
    \label{fig:attention}
\end{figure}

I use a neural network with an attention layer as an additional classifier for the tweet classification task.
The architecture is based on the ones by \citet{yang2016hierarchical} and \citet{sun2020understanding}.
Similar architectures have been used for offensive language detection by \citet{chakrabarty2019pay} and \citet{risch2020offensive}, and by \citet{jain2019attentionNotExplanation} in their discussion of attention and explanation.
The main difference between my neural model architecture and the others mentioned here, is that I use a feed-forward neural network (FFNN) rather than a recurrent one (RNN).
This decision is motivated by the discussion in \autoref{sec:attention-explanation}.

Each model input instance is represented as a sequence of $T$ tokens in an embedding matrix $z \in \R^{T \times e}$.
The encoder (in this case a feed-forward neural network) uses this embedding matrix to produce an encoded representation $h \in \R^{T \times m}$.

This representation can be compared to a \textit{context vector} $v \in \R^m$ via a similarity function $\phi$ to get a distribution of attention weights $\alpha \in \R^T$, one per input token representation:%
\footnote{\citet{yang2016hierarchical} introduce an intermediary layer between the encoder and the attention: $u = tanh(W h + b)$, and compare $u$ (rather than $h$) to $v$ in \autoref{eq:alpha}. \citet{sun2020understanding} and \citet{jain2019attentionNotExplanation} omit this additional step and proceed as outlined above.}

\begin{equation}
    \alpha = softmax(\phi(h, v))
    \label{eq:alpha}
\end{equation}

These attention weights are what I analyze in \autoref{sec:tweets-method-attn}.

The context vector is randomly initialized and not connected to the encoder or decoder output.
However, it plays a similar role to the \textit{query} in sequence-to-sequence models with attention or in self-attention%
\footnote{\citet{chakrabarty2019pay} found that attention based on context vectors generally yields better results for offensive language detection tasks than self-attention.}
layers, where that is a representation of the previous timestep by the decoder or encoder, respectively \citep{bahdanau2014jointly,vaswani2017attention-all}.%
\footnote{\citet{bahdanau2014jointly} base the query for the first timestep of a sequence on part of the hidden encoder representation for that same timestep.}

There are several different similarity functions that are widely used for attention architectures.
I use the similarity function for \textit{scaled}%
\footnote{Normalizing the matrix product based on the hidden layer size before applying the softmax function yields less extreme softmax values.
The closer the output of the softmax function is to 0 or 1, the smaller the gradients get, making it harder to efficiently update the model weights during training.}
\textit{dot-product attention}, as introduced by \citet{vaswani2017attention-all}:
\begin{equation}
    \phi(h, v) = \frac{h v}{\sqrt{m}}
    \label{eq:phi}
\end{equation}

The attention weights are used to produce a representation of the input sequence wherein the individual token representations are weighted by the attention distribution:

\begin{equation}
    h_\alpha = \sum_{t=1}^T \alpha_t \cdot h_t
    \label{eq:halpha}
\end{equation}

This attention output $h_\alpha$ is then passed to a final network layer (decoder) that generates the predicted label distribution.


\subsection{Attention weight entropy}

Where LIME's loss function limits the number of features per utterance that receive non-zero importance scores, there is no such restriction when the attention weights are calculated.
I calculate the \textit{entropy} of each utterance's attention weight vector to determine how informative this attention distribution is:

\begin{equation}
    \text{H}(\alpha{}) = - \sum_{t=1}^T \alpha_t \text{log}(\alpha_t)
\end{equation}

In the most uninformative case, where adding the attention layer does not have an impact on the decoder input, each entry within $\alpha$ is $\frac{1}{T}$. 
This gives the upper bound for the entropy: $-\text{log}(T)$.

If an utterance contains only one relevant token (per the attention distribution), the attention weight vector is one-hot encoded and the corresponding entropy is $\text{log}(1) = 0$, which is the lower bound for the potential entropy scores.